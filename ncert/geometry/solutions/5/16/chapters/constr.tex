
%\renewcommand{\thefigure}{\theenumi.\arabic{figure}}
\begin{figure}[!ht]
\centering
\resizebox{\columnwidth}{!}{\begin{tikzpicture}[scale = 0.5,>=stealth,point/.style={draw,circle,fill=black, inner sep=0.5pt},]
\node (O) at (0, 0)[point,label=above:$O$] {};
\node (A) at (-8.66, -5)[point,label=below:$A$] {};
\node (B) at (-4.89, -5)[point,label=below:$B$] {};
\node (C) at (4.89, -5)[point,label=below:$C$] {};
\node (D) at (8.66, -5)[point,label=below:$D$] {};
\node (M) at (0,-5)[point,label=below:$M$] {};

\draw (0,0) circle(7cm);
\draw (0,0) circle(10cm);
\draw (A) -- node[below=5pt]{}(B) -- (C) -- (D);
\draw[dotted] (O) -- node[below=5pt]{}(M);
\draw[dotted] (O) -- node[below=5pt]{}(A);
\draw[dotted] (O) -- node[below=5pt]{}(B);
\draw[dotted] (O) -- node[below=5pt]{}(C);
\draw[dotted] (O) -- node[below=5pt]{}(D);

\tkzMarkRightAngle[fill=blue!20, mark=|](O,M,B)

\end{tikzpicture}}
\caption{}
\label{fig:8.5.16_circle_latex}	
\end{figure}
%
%
%\renewcommand{\thefigure}{\theenumi}
%
\item {\em Construction: }In Fig. \ref{fig:8.5.16_circle_latex} the known parameters are
%\label{}
\\
%
%\solution From the given information, 
%$\triangle ABC$ are 
\begin{align}
\vec{O} &= \myvec{0\\0} ,
\\
\vec{A} &= \myvec{r\\0} ,
\label{eq:8.5.16_constr_a}
\\
 \vec{B} &= \myvec{-r\\0} 
\label{eq:8.5.16_constr_b}
\\
\vec{C}&= r\myvec{\cos\theta\\\sin\theta}
\label{eq:8.5.16_constr_cgen}
\end{align}
%
Let 
\begin{align}
\vec{D} & = r\myvec{\cos\theta_1\\\sin\theta_1} 
\label{eq:8.5.16_constr_dgen}
\end{align}
From the given information,
\begin{align}
 \norm{\vec{D} - \vec{C}} &= r
\\
\implies \brak{\vec{D} - \vec{C}}^T\brak{\vec{D} - \vec{C}} &= r^2\\
 \label{eq:8.5.16_dist_formula}
\\
\implies  \norm{D} ^2 - 2\vec{D}^T\vec{C} +  \norm{C}^2 &=r^2 
\end{align}
In the above, 
\begin{align}
\because \norm{D} &= \norm{\vec{C}} = r,
\\
\frac{\vec{D}^T\vec{C}}{\norm{D}\norm{C}}  &= \frac{1}{2}
\\
\implies \cos \brak{\theta_1-\theta} &= 60 \degree
\label{eq:8.5.16_theta_diff}
\end{align}
using the definition of the inner product.  $\because \theta$ is known, we get $\theta_1$ from \eqref{eq:8.5.16_theta_diff}
and $\vec{D}$ from 
\eqref{eq:8.5.16_constr_dgen}. 
%
\subitem Thus,
\begin{align}
BD: \vec{x} &= \vec{B} + \lambda_1 \brak{\vec{B}-\vec{D}}
\\
AC: \vec{x} &= \vec{A} + \lambda_2 \brak{\vec{A}-\vec{C}}
\end{align}
%
which can be used to obtain $\vec{E}$.

