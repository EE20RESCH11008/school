\begin{figure}[!ht]
\centering
\resizebox{\columnwidth}{!}{	\begin{tikzpicture}
[scale=2,>=stealth,point/.style={draw,circle,fill = black,inner sep=0.5pt},]

%Triangle sides
\def\a{5}
\def\b{6}
\def\c{4}
\def\R{3.023715784073818}

%Coordinates of A
%\def\p{0.5}
\def\p{((\a^2+\c^2-\b^2)/(2*\a))}
\def\q{{sqrt(\c^2-\p^2)}}

\def\x{((\a^2+\b^2-\c^2)/(2*\a))}
\def\y{{sqrt(\b^2-\x^2)}}

%Labeling points
%\node (A) at ($({((\a^2+\c^2-\b^2)/(2*\a))},{sqrt(\c^2-\x^2)} )$)[point,label=above right:$A$] {};
\node (A) at ($({((\a^2+\c^2-\b^2)/(2*\a))},{sqrt(\c^2-\p^2)} )$)[point,label=below left:$A$] {};
\node (B) at (0, 0)[point,label=below left:$B$] {};
\node (C) at (\a, 0)[point,label=below right:$C$] {};
\node (D) at ($({((\a^2+\b^2-\c^2)/(2*\a))},{sqrt(\b^2-\x^2)} )$)[point,label=below left:$D$] {};
%\node (E) at ($({(\a)}, {((\a)/((\a^2+\b^2-\c^2)/(2*\a))*(sqrt(\c^2-\p^2))})$)[point,label=below right:$E$] {};
%Circumcentre

\node (O) at (2.5,1.70084013)[point,label=above right:$O$] {};

%Drawing triangle ABC
\draw (A) -- node[left] {$\textrm{c}$} (B) -- node[below] {$\textrm{a}$} (C) -- node[above,yshift=2mm] {$\textrm{b}$} (A);

\draw (D) -- node[left] {$\textrm{e}$} (B);
\draw (D) -- node[left] {$\textrm{d}$} (C);

%\draw[dashed] (E) -- node[left] {$\textrm{}$} (B);
%\draw[dashed] (E) -- node[left] {$\textrm{}$} (C);
%Drawing OA, OB, OC
%\draw (O) -- node[left] {$\textrm{R}$} (A);
%\draw (O) -- node[below] {$\textrm{R}$} (B);
%\draw (O) -- node[below] {$\textrm{R}$} (C);
\draw (O) circle (\R);

%\tkzMarkAngle[fill=blue!50,size=.3](C,B,A)
%\tkzMarkAngle[fill=blue!50,size=.3](O,C,B)


%\tkzMarkAngle[fill=red!50](O,A,C)
\tkzMarkAngle[fill=red!50,size=.3](B,D,C)
%\tkzMarkAngle[fill=red!50,size=.3](B,E,C)

\tkzMarkAngle[fill=orange!50,size=.3](B,A,C)
%\tkzMarkAngle[fill=orange!50,size=.3](O,B,A)
\tkzMarkAngle[fill=red!50,size=.3](B,D,C)
\tkzMarkAngle[fill=red!50,size=.3](C,B,D)
\tkzMarkAngle[fill=red!50,size=.3](D,C,B)
%\tkzLabelAngle[pos=0.5](A,C,B){$\theta_1$}
%\tkzLabelAngle[pos=0.5](O,B,C){$\theta_1$}
\tkzLabelAngle[pos=0.5](B,D,C){$\theta_2$}
%\tkzLabelAngle[pos=0.5](B,E,C){$\theta_3$}

\tkzLabelAngle[pos=0.5](B,A,C){$\theta_1$}
%\tkzLabelAngle[pos=1.5](O,C,A){$\theta_3$}
\tkzLabelAngle[pos=0.5](C,B,D){$\beta_2$}
\tkzLabelAngle[pos=0.5](D,C,B){$\alpha_2$}

	
	\end{tikzpicture}
	}
\caption{ }
\label{fig:8.5.13_C_circle}	
\end{figure}
%
\item {\em Construction: }See Fig. \ref{fig:8.5.13_C_circle}.  The input parameters are
% 
\begin{align}
\vec{A} &= \myvec{p \\ q}
\\
\vec{B} &= \myvec{0 \\ 0}
\\
\vec{C} &= \myvec{a\\ 0} 
\end{align}
\subitem $\theta_1$ is obtained using
%
\begin{align}
\cos{\theta_1} = \frac{\brak{\vec{A}-\vec{B}}^T\brak{\vec{A}-\vec{C}}}{\norm{\vec{A}-\vec{B}}\norm{\vec{A}-\vec{C}}}
\end{align}
\subitem Let 
\begin{align}
\vec{D} &= b\myvec{\cos \beta_2\\ b \sin \beta_2} 
\label{eq:8.5.13_D}
\end{align}
From the given information, $\theta_2 = \theta_1$
\begin{align}
\implies \frac{b}{\sin\brak{\theta_2 + \beta_2}} = \frac{a}{\sin \theta_2}
\label{eq:8.5.13_b}
\end{align}
%
Choosing an appropriate value of $\beta_2$, $b$ and $\vec{D}$ can be obtained from \eqref{eq:8.5.13_D} and \eqref{eq:8.5.13_b} respectively.
\subitem The circumcircle of $\triangle ABC$ can then be drawn and it can be verified that $\vec{D}$ lies on it.
