\item {Proof: } Let 
\begin{align}
\vec{O} = \myvec{0\\0}
\end{align}
%
be the circumcentre of $\triangle ABC$ and let $r$ be the radius.  Assuming that
\begin{align}
\label{eq:8.5.13_A}
\vec{A} &= r\myvec{\cos \theta_1\\\sin\theta_1},
\vec{B} = r\myvec{\cos \theta_2\\\sin\theta_2}
\\
\vec{C} &= r\myvec{\cos \theta_3\\\sin\theta_3},
\vec{D} = k\myvec{\cos \theta_4\\\sin\theta_4}
\label{eq:8.5.13_D}
\end{align}
in Fig. \ref{fig:8.5.13_C_circle}, from the given information
\begin{align}
 \frac{\brak{\vec{A}-\vec{B}}^T\brak{\vec{A}-\vec{C}}}{\norm{\vec{A}-\vec{B}}\norm{\vec{A}-\vec{C}}} = 
 \frac{\brak{\vec{D}-\vec{B}}^T\brak{\vec{D}-\vec{C}}}{\norm{\vec{D}-\vec{B}}\norm{\vec{D}-\vec{C}}} 
\label{eq:8.5.13_inner}
\end{align}
\begin{multline}
\because \brak{\vec{A}-\vec{B}}^T\brak{\vec{A}-\vec{C}} = \norm{\vec{A}}^2 - \vec{A}^T\vec{B}
\\
- \vec{B}^T\vec{A}+ \vec{B}^T\vec{C}
\label{eq:8.5.13_inner_expand}
\end{multline}
from \eqref{eq:8.5.13_A}-\eqref{eq:8.5.13_D}, \eqref{eq:8.5.13_inner_expand} can be expressed as
\begin{multline}
r^2\lsbrak{1 - \cos \brak{\theta_1-\theta_2} }
\\
\rsbrak{-  \cos \brak{\theta_1-\theta_3}+ \cos \brak{\theta_2-\theta_3}}
\label{eq:8.5.13_abc_inner}
\end{multline}
%
Similarly, 
\begin{multline}
 \brak{\vec{D}-\vec{B}}^T\brak{\vec{D}-\vec{C}} = \norm{\vec{D}}^2 - \vec{D}^T\vec{B}
\\
- \vec{B}^T\vec{D}+ \vec{B}^T\vec{C}
\label{eq:8.5.13_inner_expand}
\end{multline}



