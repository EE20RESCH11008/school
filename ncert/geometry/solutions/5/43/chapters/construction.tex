
\begin{figure}[!ht]
\centering
\resizebox{\columnwidth}{!}{\begin{tikzpicture}[scale=1.5,>=stealth,point/.style={draw,circle,fill = black,inner sep=0.5pt},]
      
%Labeling points
\node (A) at (0, 0)[point,label=below left:$A$] {};
\node (B) at (5, 0)[point,label=below right:$B$] {};
\node (C) at (3.9, 3)[point,label=above right:$C$] {};
\node (D) at (1.1, 3)[point,label=above left:$D$] {};
\node (P1) at (1.1, 0)[point,label=below right:$P_1$] {};
\node (P2) at (3.9, 0)[point,label=below left:$P_2$] {};
\node (O) at (2.5, 0.79)[point,label=below:$O$]{};

%Drawing quad ABCD
\draw (A) -- node[below=6pt]{$b$}(B) -- (C) -- (D) -- (A);
\draw[dotted] (D) -- node[right = 7pt]{$h$}(P1)(P2)--(C);
\draw[dotted] (O) circle(2.62);
%marking line segment
\tkzMarkSegments[mark=|,size=6pt](A,D C,B)
\tkzMarkSegments[mark=s||,size=6pt](P1,D C,P2)

%marking angles
\tkzMarkAngle[fill=orange!40,size=0.5cm,mark=](P1,A,D)
\tkzMarkAngle[fill=orange!40,size=0.5cm,mark=](D,C,B)
\tkzMarkRightAngle[fill=blue!20](D,P1,A)
\tkzMarkRightAngle[fill=blue!20](C,P2,B)
\tkzLabelAngle[pos=0.75](P1,A,D){$\theta$}
\tkzLabelAngle[pos=0.75](D,C,B){$\alpha$}

\end{tikzpicture} }
\caption{}
\label{fig:8.5.43_trapezium}	
\end{figure}
%
\item {\em Construction: }See Fig. \ref{fig:8.5.43_trapezium}
The input parameters are
\begin{align}
\vec{A} &=\myvec{0\\0},
\vec{B} &= \myvec{b\\0}, \label{eq:8.5.43_constr_b}
\\
\vec{C} &= \myvec{b - h\cot{\theta}\\h}\label{eq:8.5.43_constr_c}\\ 
\vec{D} &= h\myvec{\cot{\theta}\\ 1}\label{eq:8.5.43_constr_d}	\end{align}
%
which are sufficient to draw the trapezium.  The circumcircle of $\triangle ABC$ is then drawn.  This circle passes through $\vec{D}$.
