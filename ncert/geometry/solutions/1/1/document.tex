\documentclass[journal,12pt,twocolumn]{IEEEtran}
%
\def\inputGnumericTable{}
\usepackage{setspace}
\usepackage{gensymb}
%\doublespacing
\singlespacing

%\usepackage{graphicx}
%\usepackage{amssymb}
%\usepackage{relsize}
\usepackage[cmex10]{amsmath}
%\usepackage{amsthm}
%\interdisplaylinepenalty=2500
%\savesymbol{iint}
%\usepackage{txfonts}
%\restoresymbol{TXF}{iint}
%\usepackage{wasysym}
\usepackage{amsthm}
%\usepackage{iithtlc}
\usepackage{mathrsfs}
\usepackage{txfonts}
\usepackage{stfloats}
\usepackage{bm}
\usepackage{cite}
\usepackage{cases}
\usepackage{subfig}
%\usepackage{xtab}
\usepackage{longtable}
\usepackage{multirow}
%\usepackage{algorithm}
%\usepackage{algpseudocode}
\usepackage{enumitem}
\usepackage{mathtools}
\usepackage{steinmetz}
\usepackage{tikz}
\usepackage{circuitikz}
\usepackage{verbatim}
\usepackage{tfrupee}
\usepackage[breaklinks=true]{hyperref}
%\usepackage{stmaryrd}
\usepackage{tkz-euclide} % loads  TikZ and tkz-base
%\usetkzobj{all}
\usetikzlibrary{calc,math}
\usepackage{listings}
    \usepackage{color}                                            %%
    \usepackage{array}                                            %%
    \usepackage{longtable}                                        %%
    \usepackage{calc}                                             %%
    \usepackage{multirow}                                         %%
    \usepackage{hhline}                                           %%
    \usepackage{ifthen}                                           %%
  %optionally (for landscape tables embedded in another document): %%
    \usepackage{lscape}     
\usepackage{multicol}
\usepackage{chngcntr}
%\usepackage{enumerate}

%\usepackage{wasysym}
%\newcounter{MYtempeqncnt}
\DeclareMathOperator*{\Res}{Res}
%\renewcommand{\baselinestretch}{2}
\renewcommand\thesection{\arabic{section}}
\renewcommand\thesubsection{\thesection.\arabic{subsection}}
\renewcommand\thesubsubsection{\thesubsection.\arabic{subsubsection}}

\renewcommand\thesectiondis{\arabic{section}}
\renewcommand\thesubsectiondis{\thesectiondis.\arabic{subsection}}
\renewcommand\thesubsubsectiondis{\thesubsectiondis.\arabic{subsubsection}}

% correct bad hyphenation here
\hyphenation{op-tical net-works semi-conduc-tor}
\def\inputGnumericTable{}                                 %%

\lstset{
%language=C,
frame=single, 
breaklines=true,
columns=fullflexible
}
%\lstset{
%language=tex,
%frame=single, 
%breaklines=true
%}

\begin{document}
%


\newtheorem{theorem}{Theorem}[section]
\newtheorem{problem}{Problem}
\newtheorem{proposition}{Proposition}[section]
\newtheorem{lemma}{Lemma}[section]
\newtheorem{corollary}[theorem]{Corollary}
\newtheorem{example}{Example}[section]
\newtheorem{definition}[problem]{Definition}
%\newtheorem{thm}{Theorem}[section] 
%\newtheorem{defn}[thm]{Definition}
%\newtheorem{algorithm}{Algorithm}[section]
%\newtheorem{cor}{Corollary}
\newcommand{\BEQA}{\begin{eqnarray}}
\newcommand{\EEQA}{\end{eqnarray}}
\newcommand{\define}{\stackrel{\triangle}{=}}
\bibliographystyle{IEEEtran}
%\bibliographystyle{ieeetr}
\providecommand{\mbf}{\mathbf}
\providecommand{\pr}[1]{\ensuremath{\Pr\left(#1\right)}}
\providecommand{\qfunc}[1]{\ensuremath{Q\left(#1\right)}}
\providecommand{\sbrak}[1]{\ensuremath{{}\left[#1\right]}}
\providecommand{\lsbrak}[1]{\ensuremath{{}\left[#1\right.}}
\providecommand{\rsbrak}[1]{\ensuremath{{}\left.#1\right]}}
\providecommand{\brak}[1]{\ensuremath{\left(#1\right)}}
\providecommand{\lbrak}[1]{\ensuremath{\left(#1\right.}}
\providecommand{\rbrak}[1]{\ensuremath{\left.#1\right)}}
\providecommand{\cbrak}[1]{\ensuremath{\left\{#1\right\}}}
\providecommand{\lcbrak}[1]{\ensuremath{\left\{#1\right.}}
\providecommand{\rcbrak}[1]{\ensuremath{\left.#1\right\}}}
\theoremstyle{remark}
\newtheorem{rem}{Remark}
\newcommand{\sgn}{\mathop{\mathrm{sgn}}}
\providecommand{\abs}[1]{\left\vert#1\right\vert}
\providecommand{\res}[1]{\Res\displaylimits_{#1}} 
\providecommand{\norm}[1]{\left\lVert#1\right\rVert}
%\providecommand{\norm}[1]{\lVert#1\rVert}
\providecommand{\mtx}[1]{\mathbf{#1}}
\providecommand{\mean}[1]{E\left[ #1 \right]}
\providecommand{\fourier}{\overset{\mathcal{F}}{ \rightleftharpoons}}
%\providecommand{\hilbert}{\overset{\mathcal{H}}{ \rightleftharpoons}}
\providecommand{\system}{\overset{\mathcal{H}}{ \longleftrightarrow}}
	%\newcommand{\solution}[2]{\textbf{Solution:}{#1}}
\newcommand{\solution}{\noindent \textbf{Solution: }}
\newcommand{\cosec}{\,\text{cosec}\,}
\providecommand{\dec}[2]{\ensuremath{\overset{#1}{\underset{#2}{\gtrless}}}}
\newcommand{\myvec}[1]{\ensuremath{\begin{pmatrix}#1\end{pmatrix}}}
\newcommand{\mydet}[1]{\ensuremath{\begin{vmatrix}#1\end{vmatrix}}}
%\numberwithin{equation}{section}
\numberwithin{equation}{subsection}
%\numberwithin{problem}{section}
%\numberwithin{definition}{section}
\makeatletter
\@addtoreset{figure}{problem}
\makeatother
\let\StandardTheFigure\thefigure
\let\vec\mathbf
%\renewcommand{\thefigure}{\theproblem.\arabic{figure}}
%\renewcommand{\thefigure}{\theproblem}
%\setlist[enumerate,1]{before=\renewcommand\theequation{\theenumi.\arabic{equation}}
%\counterwithin{equation}{enumi}
%\renewcommand{\theequation}{\arabic{subsection}.\arabic{equation}}
\def\putbox#1#2#3{\makebox[0in][l]{\makebox[#1][l]{}\raisebox{\baselineskip}[0in][0in]{\raisebox{#2}[0in][0in]{#3}}}}
     \def\rightbox#1{\makebox[0in][r]{#1}}
     \def\centbox#1{\makebox[0in]{#1}}
     \def\topbox#1{\raisebox{-\baselineskip}[0in][0in]{#1}}
     \def\midbox#1{\raisebox{-0.5\baselineskip}[0in][0in]{#1}}
\vspace{3cm}
\title{Question 1 Exercise(8.1)}
\author{Srihari S}


\maketitle
\begin{abstract}
A question based on properties of triangles.
\end{abstract}
Download all python codes from 
%
\begin{lstlisting}
svn co https://github.com/Srihari123456/Summer-2020/tree/master/geometry/triangle/codes
\end{lstlisting}
Download all \LaTeX{}-Tikz codes from 
%
\begin{lstlisting}
svn co https://github.com/Srihari123456/Summer-2020/tree/master/geometry/triangle/figs
\end{lstlisting}

\section{\textbf{Question}}
Find p(0), p(1), p(2) for each of the following polynomials
\begin{align}
\brak{a} p(y) = y^{2}
\brak{b} p(x) = \brak{x-1}\brak{x+1}
\end{align}

%\newline
\section{\textbf{Construction}}
\begin{figure}[!ht]
\centering
\resizebox{\columnwidth}{!}{\begin{tikzpicture}
[scale=2,>=stealth,point/.style={draw,circle,fill = black,inner sep=0.5pt},]

%Triangle sides
\def\a{5}
\def\b{6}
\def\c{4}
 
%Coordinates of A
\def\p{0.5}
\def\q{{sqrt(\c^2-\p^2)}}

%Labeling points
\node (A) at (\p,\q)[point,label=above right:$A$] {};
\node (B) at (0, 0)[point,label=below left:$B$] {};
\node (C) at (\a, 0)[point,label=below right:$C$] {};
\node (O) at (2,1.5)[point,label=below:$O$]{};
%Foot of median

\node (D) at ($(A)!0.5!(O)$)[point,label=below:$D$] {};
\node (E) at ($(A)!0.5!(B)$)[point,label=left:$E$] {};
\node (F) at ($(C)!0.5!(A)$)[point,label=right:$F$] {};

%Drawing triangle ABC
\draw (A) -- node[left, xshift=-5mm,yshift=5mm] {$\textrm{c}$} (B) -- node[below, yshift=-5mm] {$\textrm{a}$} (C) -- node[above right,xshift=2mm,yshift=5mm] {$\textrm{b}$} (A);
\draw (A) -- (D) -- (O);
%Drawing medians BE and CF
\draw (D) -- (E);
\draw (D) -- (F);
\draw (O) -- (B);
\draw (O) -- (C);
%Drawing EF
\draw (E) -- (F);

%Labeling sides
%\node [right] at ($(A)!0.5!(E)$) {$\frac{b}{2}$};
%\node [right] at ($(C)!0.5!(E)$) {$\frac{b}{2}$};
%\node [left] at ($(B)!0.5!(F)$) {$\frac{c}{2}$};
%\node [left] at ($(A)!0.5!(F)$) {$\frac{c}{2}$};




%Angles
\tkzMarkAngle[size=.3](D,E,A)
\tkzMarkAngle[size=.3](O,B,A)
%
\tkzMarkAngle[size=.7](A,F,D)
\tkzMarkAngle[size=.7](A,C,O)
%%
\tkzMarkAngle[size=.2](A,D,E)
\tkzMarkAngle[size=.2](A,O,B)
%%
\tkzMarkAngle[size=.2](F,D,A)
\tkzMarkAngle[size=.2](C,O,A)
%
%\tkzMarkAngle[size=.3](E,A,D)
%\tkzMarkAngle[size=.3](D,A,F)

\begin{comment}
%Angles
\tkzMarkAngle[fill=green!60,size=.3](D,E,A)
\tkzMarkAngle[fill=green!60,size=.3](O,B,A)
%
\tkzMarkAngle[fill=red!60,size=.5](A,F,D)
\tkzMarkAngle[fill=red!60,size=.5](A,C,O)
%%
\tkzMarkAngle[fill=yellow!60,size=.2](A,D,E)
\tkzMarkAngle[fill=yellow!60,size=.2](A,O,B)
%%
\tkzMarkAngle[fill=orange!60,size=.2](F,D,A)
\tkzMarkAngle[fill=orange!60,size=.2](C,O,A)
%
\tkzMarkAngle[fill=blue!60,size=.3](E,A,D)
\tkzMarkAngle[fill=blue!60,size=.3](D,A,F)
\end{comment}
\end{tikzpicture}
}
\caption{}
\label{fig:8.1.1_similar}	
\end{figure}


\item {\em Construction: } See Fig. \ref{fig:8.1.1_similar}.
The input parameters are
\begin{multline}
 \vec{B}= \myvec{0\\0},
\vec{C}=\myvec{a\\0},
\vec{A}=a\myvec{\cos 60\degree\\ \sin 60\degree}
\end{multline}

\section{\textbf{Solution}}
\item {\em Proof: } Using the cosine formula,
%
\begin{align}
\cos \phase{ABC} &= \frac{a^2 +a^2 - a^2}{2a^2}
\\
&= \frac{1}{2}
\\
\implies  \phase{ABC} &= 60 \degree
\end{align}
%



\end{document}
