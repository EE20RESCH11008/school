
%\renewcommand{\thefigure}{\theenumi.\arabic{figure}}
\begin{figure}[!ht]
\centering
\resizebox{\columnwidth}{!}{\begin{tikzpicture}[scale = 1.5,>=stealth,point/.style={draw,circle,fill=black, inner sep=0.5pt},]

\node (Q) at (0, 0)[point,label=below left:$Q$] {};
\node (R) at (6, 0)[point,label=below right:$R$] {};
\node (P) at (2.25, 3.307189138830738)[point,label=above right:$P$] {};
\node (S) at (8/3, 0)[point,label=below:$S$] {};
\node (G) at (4/3, 0)[label=below:$y$] {};

\draw (P) -- node[below=5pt]{}(Q) -- (R) -- (P) -- (S);

\tkzMarkAngle[fill=green!20, mark=|](Q,P,S)
\tkzMarkAngle[fill=green!20, mark=|](S,P,R)

\end{tikzpicture}}
\caption{$\triangle ABC$ and $\triangle PQR$ by Latex-Tikz}
\label{fig:8.1.36_triangle_latex}	
\end{figure}
%
%
%\renewcommand{\thefigure}{\theenumi}
%

\item {\em Construction: }The coordinates of the various points of triangle ABC in Fig. \ref{fig:8.1.36_triangle_latex} are
%\label{}
%
\begin{align}
\vec{B} &= \myvec{0\\0} ,
\label{eq:8.1.36_constr_b}
\\
 \vec{C} &= \myvec{a\\0}, 
\label{eq:8.1.36_constr_c}
\end{align}

$\because \vec{M}$ is the midpoint of $BC$,
\begin{align}
\vec{M}= \frac{\vec{B}+\vec{C}}{2} =\myvec{a/2\\0},
\label{eq:8.1.36_constr_m}
\end{align}
%
$\triangle PQR$ is a horizontal translation of $\triangle ABC$.  Hence, if 
\begin{align}
\vec{Q}= \myvec{q\\0},
\label{eq:8.1.36_constr_q}
\end{align}
\begin{align}
\vec{P}= \vec{A} + \vec{Q}
\\
\vec{R}= \vec{C} + \vec{Q}
\end{align}

%



