\renewcommand{\theequation}{\theenumi}
\begin{enumerate}[label=\arabic*.,ref=\thesubsection.\theenumi]
\numberwithin{equation}{enumi}
    \item Solve $30x < 200$ when
    \begin{enumerate} 
    \item  x is a natural number,
    \item x is an integer.
\end{enumerate}
\solution From the given information, 
\begin{align}
30x < 200 \implies x < \frac{20}{3}
\label{eq:lineq_nat}
\end{align}
If $x$ is a natural number, $x \in \cbrak{1, 2, 3, 4, 5, 6}$. If $x$ is an integer, then the solution set includes 0 as well as all negative integers.
    \item Solve $5x-3 < 3x+1$ when
    \begin{enumerate} 
\item  x is an integer,
    \item x is a real number.
\end{enumerate}
\solution 
\begin{align}
5x-3 < 3x+1 \implies x < 2
\label{eq:lineq_real}
\end{align}
%
If $x$ is real, then $x \in \brak{-\infty, 2}$. 
%Fig.  provides a graphical solution using the following python code
%\begin{lstlisting}
%\end{lstlisting}
    \item Solve the following system of linear inequalities graphically.
\begin{align}
\label{eq:line_two_ineq}
\begin{split}
    x+y &\geq 5
\\
    x-y &\leq 3
\end{split}
\end{align}
\solution  Let $u_1 \ge 0, u_2 \ge 0$.  This may be expressed as
\begin{align}
\vec{u} = \myvec{u_1\\u_2}\succeq \vec{0}
\end{align}
%
\eqref{eq:line_two_ineq} can then be expressed as
\begin{align}
\begin{split}
    x+y &\geq 5
\\
    -x+y &\geq -3
\end{split}
%
\\
\implies 
\myvec{1 & 1 \\ -1 & 1}\vec{x}  &\succeq \myvec{5\\-3}
\\
\myvec{1 & 1 \\ -1 & 1}\vec{x}  -\vec{u}&=\myvec{5\\-3}
\\
\text{or, }
\myvec{1 & 1 \\ -1 & 1}\vec{x} &= \myvec{5\\-3} +\vec{u}
\end{align}
%
resulting in 
\begin{align}
\vec{x} &= \myvec{1 & 1 \\ -1 & 1}^{-1}\myvec{5\\-3} +\myvec{1 & 1 \\ -1 & 1}^{-1}\vec{u}
\\
\text{or, } \vec{x} &= \myvec{4\\1} +\frac{1}{2}\myvec{1 & -1 \\ 1 & 1}\vec{u}
\end{align}
%
after obtaining the  inverse.
%
 Fig. \ref{fig:line_ineq} generated using the following python code shows the region satisfying \eqref{eq:line_two_ineq}

\begin{lstlisting}
codes/line/line_ineq.py
\end{lstlisting}
%
\begin{figure}[!ht]
\includegraphics[width=\columnwidth]{./line/figs/line_ineq.eps}
\caption{}
\label{fig:line_ineq}
\end{figure}
%
\item Solve 
\begin{align}
\begin{split}
2x+y \geq 4
\\ 
x+y \leq 3
\\ 
2x-3y \leq 6
\end{split}
\label{eq:line_mult_ineq}
\end{align}
%
\\
\solution  Fig. \ref{fig:line_ineq_mult} generated using the following python code shows the region satisfying \eqref{eq:line_mult_ineq}

\begin{lstlisting}
codes/line/line_ineq_mult.py
\end{lstlisting}
%
\begin{figure}[!ht]
\includegraphics[width=\columnwidth]{./line/figs/line_ineq_mult.eps}
\caption{}
\label{fig:line_ineq_mult}
\end{figure}
%
\item   Solve    $x+y < 5$ graphically.
\\
\solution  See Fig.  generated using the following python code
\begin{lstlisting}
\end{lstlisting}
%
    \item Solve 
\begin{align}
\myvec{3 & 2 \\ 1 & 4 \\ 1 & 0 \\ 0 & -1 \\ -1 & 0} \vec{x}\preceq \myvec{150\\80\\15\\0\\0}
%3x+2y \leq 150
%\\ 
%x+4y \leq 80
%\\ 
%x \leq 15
%\\ 
%y \geq 0
%\\
%x \geq 0 
\end{align}
%
\solution  
See Fig.  generated using the following python code
\begin{lstlisting}
\end{lstlisting}
   
    \end{enumerate}
