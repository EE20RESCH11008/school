let X be the random variable representing the result when a die is thrown
\begin{align}
X \epsilon \{0,1,2,3,4,5,6\}
\end{align}
All results are equally likely in a fair die. Hence 
\begin{align}
\pr{X = 6} = \frac{1}{6}
\\
\pr{X \neq 6} = \frac{5}{6}
\end{align}
For A to win the game at k-th throw, A should throw a 6 in the k-th throw and both A and B must not throw a six in the preceding \brak{k-1} throws.So probability of A winning after k thows is given as
\begin{align}
\pr{A_k}={\brak{\frac{5}{6}}}^{k-1}\frac{1}{6}
\\
k \epsilon \{1,2....\infty\}
\end{align}
So the total probability of A winning is given as
\begin{align}
\pr{A}=\sum_{k=0}^{\infty}\pr{A_k}
\\
\pr{A}=\frac{1}{6}+{\brak{\frac{5}{6}}}^{2}\frac{1}{6}+{\brak{\frac{5}{6}}}^{4}\frac{1}{6}+....
\\
\pr{A}=\frac{1}{6}\times\brak{1+{\brak{\frac{5}{6}}}^{2}+{\frac{5}{6}}^{4}+....}
\\
\pr{A}=\frac{1}{6}\times\frac{1}{1-{\brak{\frac{5}{6}}}^2}
\\
\pr{A}=\frac{6}{11}
\end{align}
Similarly probability of B winning is given as
\begin{align}
\pr{B}=\frac{5}{6}\times\frac{1}{6}+{\brak{\frac{5}{6}}}^{3}\frac{1}{6}+{\brak{\frac{5}{6}}}^{5}\frac{1}{6}+....
\\
\pr{B}=\frac{1}{6}\times\frac{5}{6}\times\brak{1+{\brak{\frac{5}{6}}}^{2}+{\brak{\frac{5}{6}}}^{4}+....}
\\
\pr{B}=\frac{5}{11}
\end{align}
The python code for the above problem is,
\begin{lstlisting}
./prob/codes/exam46.py
\end{lstlisting}
In the above code 1000000 random outputs of a die are generated for A and B each.The probabilities are calculated using the total number of times A gets a six first and the total number of times B get a six first.
