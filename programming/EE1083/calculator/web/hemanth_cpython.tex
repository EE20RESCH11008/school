\documentclass[journal,12pt,twocolumn]{IEEEtran}
%
\makeatletter
\@addtoreset{figure}{problem}
\makeatother
\usepackage{setspace}
\usepackage{gensymb}
\usepackage{xcolor}
\usepackage{caption}
%\usepackage{subcaption}
%\doublespacing
\singlespacing

%\usepackage{graphicx}
%\usepackage{amssymb}
%\usepackage{relsize}
\usepackage[cmex10]{amsmath}
\usepackage{mathtools}
%\usepackage{amsthm}
%\interdisplaylinepenalty=2500
%\savesymbol{iint}
%\usepackage{txfonts}
%\restoresymbol{TXF}{iint}
%\usepackage{wasysym}
\usepackage{amsthm}
\usepackage{mathrsfs}
\usepackage{txfonts}
\usepackage{stfloats}
\usepackage{cite}
\usepackage{cases}
\usepackage{mathtools}
\usepackage{subfig}

%\usepackage{xtab}
\usepackage{longtable}
\usepackage{multirow}
%\usepackage{algorithm}
%\usepackage{algpseudocode}
\usepackage{enumitem}
\usepackage{mathtools}
\usepackage{iithtlc}
%\usepackage[framemethod=tikz]{mdframed}
\usepackage{listings}
    \usepackage[latin1]{inputenc}                                 %%
    \usepackage{color}                                            %%
    \usepackage{array}                                            %%
    \usepackage{longtable}                                        %%
    \usepackage{calc}                                             %%
    \usepackage{multirow}                                         %%
    \usepackage{hhline}                                           %%
    \usepackage{ifthen}                                           %%
  %optionally (for landscape tables embedded in another document): %%
    \usepackage{lscape}     

\usepackage[breaklinks]{hyperref}
\usepackage{url}
\def\UrlBreaks{\do\/\do-}
\usepackage{breakurl}


%\usepackage{stmaryrd}


%\usepackage{wasysym}
%\newcounter{MYtempeqncnt}
\DeclareMathOperator*{\Res}{Res}
%\renewcommand{\baselinestretch}{2}
\renewcommand\thesection{\arabic{section}}
\renewcommand\thesubsection{\thesection.\arabic{subsection}}
\renewcommand\thesubsubsection{\thesubsection.\arabic{subsubsection}}

\renewcommand\thesectiondis{\arabic{section}}
\renewcommand\thesubsectiondis{\thesectiondis.\arabic{subsection}}
\renewcommand\thesubsubsectiondis{\thesubsectiondis.\arabic{subsubsection}}

% correct bad hyphenation here
\hyphenation{op-tical net-works semi-conduc-tor}

%\lstset{
%language=C,
%frame=single, 
%breaklines=true
%}

%\lstset{
	%%basicstyle=\small\ttfamily\bfseries,
	%%numberstyle=\small\ttfamily,
	%language=Octave,
	%backgroundcolor=\color{white},
	%%frame=single,
	%%keywordstyle=\bfseries,
	%%breaklines=true,
	%%showstringspaces=false,
	%%xleftmargin=-10mm,
	%%aboveskip=-1mm,
	%%belowskip=0mm
%}

%\surroundwithmdframed[width=\columnwidth]{lstlisting}
\def\inputGnumericTable{}                                 %%
\lstset{
language=C,
frame=single, 
breaklines=true
}
 

\begin{document}
%

\theoremstyle{definition}
\newtheorem{theorem}{Theorem}[section]
\newtheorem{problem}{Problem}
\newtheorem{proposition}{Proposition}[section]
\newtheorem{lemma}{Lemma}[section]
\newtheorem{corollary}[theorem]{Corollary}
\newtheorem{example}{Example}[section]
\newtheorem{definition}{Definition}[section]
%\newtheorem{algorithm}{Algorithm}[section]
%\newtheorem{cor}{Corollary}
\newcommand{\BEQA}{\begin{eqnarray}}
\newcommand{\EEQA}{\end{eqnarray}}
\newcommand{\define}{\stackrel{\triangle}{=}}

\bibliographystyle{IEEEtran}
%\bibliographystyle{ieeetr}

\providecommand{\nCr}[2]{\,^{#1}C_{#2}} % nCr
\providecommand{\nPr}[2]{\,^{#1}P_{#2}} % nPr
\providecommand{\mbf}{\mathbf}
\providecommand{\pr}[1]{\ensuremath{\Pr\left(#1\right)}}
\providecommand{\qfunc}[1]{\ensuremath{Q\left(#1\right)}}
\providecommand{\sbrak}[1]{\ensuremath{{}\left[#1\right]}}
\providecommand{\lsbrak}[1]{\ensuremath{{}\left[#1\right.}}
\providecommand{\rsbrak}[1]{\ensuremath{{}\left.#1\right]}}
\providecommand{\brak}[1]{\ensuremath{\left(#1\right)}}
\providecommand{\lbrak}[1]{\ensuremath{\left(#1\right.}}
\providecommand{\rbrak}[1]{\ensuremath{\left.#1\right)}}
\providecommand{\cbrak}[1]{\ensuremath{\left\{#1\right\}}}
\providecommand{\lcbrak}[1]{\ensuremath{\left\{#1\right.}}
\providecommand{\rcbrak}[1]{\ensuremath{\left.#1\right\}}}
\theoremstyle{remark}
\newtheorem{rem}{Remark}
\newcommand{\sgn}{\mathop{\mathrm{sgn}}}
\providecommand{\abs}[1]{\left\vert#1\right\vert}
\providecommand{\res}[1]{\Res\displaylimits_{#1}} 
\providecommand{\norm}[1]{\lVert#1\rVert}
\providecommand{\mtx}[1]{\mathbf{#1}}
\providecommand{\mean}[1]{E\left[ #1 \right]}
\providecommand{\fourier}{\overset{\mathcal{F}}{ \rightleftharpoons}}
%\providecommand{\hilbert}{\overset{\mathcal{H}}{ \rightleftharpoons}}
\providecommand{\system}{\overset{\mathcal{H}}{ \longleftrightarrow}}
	%\newcommand{\solution}[2]{\textbf{Solution:}{#1}}
\newcommand{\solution}{\noindent \textbf{Solution: }}
\providecommand{\dec}[2]{\ensuremath{\overset{#1}{\underset{#2}{\gtrless}}}}
\DeclarePairedDelimiter{\ceil}{\lceil}{\rceil}
%\numberwithin{equation}{subsection}
\numberwithin{equation}{problem}
%\numberwithin{problem}{subsection}
%\numberwithin{definition}{subsection}
\makeatletter
\@addtoreset{figure}{problem}
\makeatother

\let\StandardTheFigure\thefigure
%\renewcommand{\thefigure}{\theproblem.\arabic{figure}}
\renewcommand{\thefigure}{\theproblem}


%\numberwithin{figure}{subsection}

%\numberwithin{equation}{subsection}
%\numberwithin{equation}{section}
\numberwithin{equation}{problem}
%\numberwithin{problem}{subsection}
%\numberwithin{problem}{section}
%%\numberwithin{definition}{subsection}
%\makeatletter
%\@addtoreset{figure}{problem}
%\makeatother
\makeatletter
\@addtoreset{table}{problem}
\makeatother

\let\StandardTheFigure\thefigure
\let\StandardTheTable\thetable
%%\renewcommand{\thefigure}{\theproblem.\arabic{figure}}
%\renewcommand{\thefigure}{\theproblem}

%%\numberwithin{figure}{section}

%%\numberwithin{figure}{subsection}



\def\putbox#1#2#3{\makebox[0in][l]{\makebox[#1][l]{}\raisebox{\baselineskip}[0in][0in]{\raisebox{#2}[0in][0in]{#3}}}}
     \def\rightbox#1{\makebox[0in][r]{#1}}
     \def\centbox#1{\makebox[0in]{#1}}
     \def\topbox#1{\raisebox{-\baselineskip}[0in][0in]{#1}}
     \def\midbox#1{\raisebox{-0.5\baselineskip}[0in][0in]{#1}}

\vspace{3cm}

\title{ 
	\logo{
Calculator: Shared C libraries in Python
	}
}



% paper title
% can use linebreaks \\ within to get better formatting as desired
%\title{Digital Band Pass Filter design}
%
%
% author names and IEEE memberships
% note positions of commas and nonbreaking spaces ( ~ ) LaTeX will not break
% a structure at a ~ so this keeps an author's name from being broken across
% two lines.
% use \thanks{} to gain access to the first footnote area
% a separate \thanks must be used for each paragraph as LaTeX2e's \thanks
% was not built to handle multiple paragraphs
%

\author{Hemanth Kumar Desineedi and G V V Sharma$^{*}$% <-this % stops a space
\thanks{*The author is with the Department
of Electrical Engineering, Indian Institute of Technology, Hyderabad
502285 India e-mail:  gadepall@iith.ac.in. All content in this manual is released under GNU GPL.  Free and open source.}% <-this % stops a space
%\thanks{J. Doe and J. Doe are with Anonymous University.}% <-this % stops a space
%\thanks{Manuscript received April 19, 2005; revised January 11, 2007.}}
}
% note the % following the last \IEEEmembership and also \thanks - 
% these prevent an unwanted space from occurring between the last author name
% and the end of the author line. i.e., if you had this:
% 
% \author{....lastname \thanks{...} \thanks{...} }
%                     ^------------^------------^----Do not want these spaces!
%
% a space would be appended to the last name and could cause every name on that
% line to be shifted left slightly. This is one of those "LaTeX things". For
% instance, "\textbf{A} \textbf{B}" will typeset as "A B" not "AB". To get
% "AB" then you have to do: "\textbf{A}\textbf{B}"
% \thanks is no different in this regard, so shield the last } of each \thanks
% that ends a line with a % and do not let a space in before the next \thanks.
% Spaces after \IEEEmembership other than the last one are OK (and needed) as
% you are supposed to have spaces between the names. For what it is worth,
% this is a minor point as most people would not even notice if the said evil
% space somehow managed to creep in.



% The paper headers
%\markboth{Journal of \LaTeX\ Class Files,~Vol.~6, No.~1, January~2007}%
%{Shell \MakeLowercase{\textit{et al.}}: Bare Demo of IEEEtran.cls for Journals}
% The only time the second header will appear is for the odd numbered pages
% after the title page when using the twoside option.
% 
% *** Note that you probably will NOT want to include the author's ***
% *** name in the headers of peer review papers.                   ***
% You can use \ifCLASSOPTIONpeerreview for conditional compilation here if
% you desire.




% If you want to put a publisher's ID mark on the page you can do it like
% this:
%\IEEEpubid{0000--0000/00\$00.00~\copyright~2007 IEEE}
% Remember, if you use this you must call \IEEEpubidadjcol in the second
% column for its text to clear the IEEEpubid mark.



% make the title area
\maketitle

\tableofcontents

\bigskip

\begin{abstract}
This manual shows how to build a calculator using Python and shared C libraries.  Through this, even 
beginners can learn how to build some simple software applications with graphical user interfaces (GUIs).


\end{abstract}
% IEEEtran.cls defaults to using nonbold math in the Abstract.
% This preserves the distinction between vectors and scalars. However,
% if the journal you are submitting to favors bold math in the abstract,
% then you can use LaTeX's standard command \boldmath at the very start
% of the abstract to achieve this. Many IEEE journals frown on math
% in the abstract anyway.

% Note that keywords are not normally used for peerreview papers.
%\begin{IEEEkeywords}
%Cooperative diversity, decode and forward, piecewise linear
%\end{IEEEkeywords}



% For peer review papers, you can put extra information on the cover
% page as needed:
% \ifCLASSOPTIONpeerreview
% \begin{center} \bfseries EDICS Category: 3-BBND \end{center}
% \fi
%
% For peerreview papers, this IEEEtran command inserts a page break and
% creates the second title. It will be ignored for other modes.
\IEEEpeerreviewmaketitle


%\newpage
%\section{Component Pin Diagrams}
%%
%\renewcommand{\theequation}{\theenumi}
\begin{enumerate}[label=\thesection.\arabic*.,ref=\thesection.\theenumi]
\numberwithin{equation}{enumi}
\item Let the medians $BE$ and $CF$ in Fig. \ref{fig:3.12.3_ch1_two_median} intersect at $O$, such that
\begin{equation}
\begin{split}
\frac{OB}{OE} &= k_1
\\
\frac{OC}{OF} &= k_2
\end{split}
\end{equation}
%Then  $k_1 = k_2 = 2$.
%
\begin{figure}[!h]
\centering
\resizebox {\columnwidth} {!} {
\begin{tikzpicture}
  [
    scale=2,
    >=stealth,
    point/.style = {draw, circle,  fill = black, inner sep = 0.5pt},
    dot/.style   = {draw, circle,  fill = black, inner sep = .2pt},
  ]
  \coordinate [point, label={below left:$B$ $(0, 0)$}] (B) at (0, 0);
    \node (A) at +(60:{2*sqrt(3)}) [point, label = above:$A$ ${(a,b)}$  ] {};
  \coordinate [point, label={below left:$(c,0)$ $C$ }] (C) at ($ (3,0) + sqrt(3)*(1,0) $);

  \draw  (A) -- (C) -- (B) -- (A);
  \node (E) at ($(A)!0.5!(C)$) [point, label = {right:$E$}]{};
  \node (F) at ($(A)!0.5!(B)$) [point, label = {left:$F$}]{};
  \path
     (B)    edge  node[sloped, anchor=center, below, text width=2.0cm] { $k_1:1$}     (E)  
	 (C)    edge  node[sloped, anchor=east, below, text width=2.0cm] { $1:k_2$}     (F);
  \node (O) at ($(B)!0.67!(E)$) [point, label = {below:$O$}]{};  
\end{tikzpicture}


}
\caption{Medians $BE$ and $CF$}
\label{fig:3.12.3_ch1_two_median}
\end{figure}
%Let the coordinates of $A$, $B$ and $C$ be $\brak{a,b}$, $\brak{0,0}$ and $\brak{c,0}$ respectively. 
Using \eqref{eq:line_section_form},
%
\begin{align}
E &= \frac{\vec{A}+\vec{B}}{2} 
\\
F &= \frac{\vec{A}+\vec{C}}{2} 
\label{eq:3.12.3_ch1_ratio_ef}
\end{align}
%
Similarly, since $O$ divides $BE$ in the ratio $k_1:1$ and $CF$ in the ratio $k_2:1$.
 %
\begin{align}
O = \frac{k_1\vec{E}+\vec{B}}{k_1+1} &=  \frac{k_2\vec{F}+\vec{C}}{k_2+1} 
\\
\implies \frac{k_1\brak{\frac{\vec{A}+\vec{B}}{2}} +B}{k_1+1} &=  \frac{k_2\brak{\frac{\vec{A}+\vec{C}}{2} }+C}{k_2+1} 
\label{eq:3.12.3_ch1_ratio_2}
\end{align}
upon substituting from \eqref{eq:3.12.3_ch1_ratio_ef}.
Simplifying \eqref{eq:3.12.3_ch1_ratio_2},
\begin{align}
\frac{k_1\brak{\vec{A}+\vec{C}} +2\vec{B}}{k_1+1} =  \frac{k_2\brak{\vec{A}+\vec{B}}+2\vec{C}}{k_2+1} 
\end{align}
which can be expressed as
\begin{multline}
\implies \sbrak{k_1\brak{k_2+1}-k_2\brak{k_1+1}}\vec{A}
\\
 +\sbrak{2\brak{k_2+1}-k_2\brak{k_1+1}}\vec{B}
\\ +  \sbrak{k_1\brak{k_2+1} -2\brak{k_1+1}}\vec{C} = 0
\end{multline}
resulting in 
\begin{align}
\vec{B} = \frac{\brak{k_1-k_2}\vec{A}+\brak{k_1k_2 -k_1 -2}}{k_1k_2 -k_2 -2}
\end{align}
%
If the above equation has a solution, then $\vec{A}, \vec{B}$ and $\vec{C}$ lie on a straight line.  Since that is not the case, the only possibility is 
\begin{align}
k_1-k_2 &= 0
\\
k_1k_2 -k_1 -2 &= 0
\\
k_1k_2 -k_2 -2 &= 0
\\
\implies k_1=k_2&=2
\end{align}
\item In Fig. \ref{fig:3.12.3_ch1_two_median},
\begin{align}
\vec{E} &=  \frac{\vec{A}+\vec{C}}{2} \quad \text{and}
\\
\vec{O}&= \frac{\vec{B}+2\vec{E}}{3}
\\
&= \frac{\vec{A}+\vec{B}+\vec{C}}{3}
\end{align}
\end{enumerate}
	

%

%\newpage

\section{Python Calculator}
\begin{problem}
\label{prob:calc}
Download the python code from \cite{tkcalc}
%\href{https://github.com/gadepall/EE1083/blob/master/calculator/codes/solution/pythonprogs/tkcalc.py}{\url{code}} from 
%\begin{lstlisting}
%https://github.com/gadepall/EE1083/blob/master/calculator/codes/solution/pythonprogs/tkcalc.py
%\end{lstlisting}
and execute it.
\end{problem}

\section{Shared Libraries in GCC}
\begin{problem}
Write a C function to multiply two given numbers. Save it in the file titled as \textbf{mul.c}
\end{problem}
\solution
\lstinputlisting{./codes/cprogs/mul.c}
\begin{problem}
Open the Terminal and go to the directory where the \textbf{mul.c} file is saved.
\end{problem}
\begin{problem}
Type the following command in the Terminal.
\end{problem}
\solution
\begin{lstlisting}
cc -fPIC -shared -o mul.so mul.c
\end{lstlisting}
Note that you will have to use the \textbf{-lm} switch for \textbf{math.h } functions.
\begin{problem}
Type the following program in \textbf{main.c}
\end{problem}
\solution
\lstinputlisting{./codes/cprogs/main.c}
\begin{problem}
Run the above program
\end{problem}
\solution
\begin{lstlisting}
gcc main.c mul.so -Wl,-rpath=$(pwd)
./a.out
\end{lstlisting}
The advantange of using \textbf{mul.so} is that the multiplication function needs to be compiled only once.  It can then be
used in any C program.
\begin{problem}
Repeat the above exercises for adding two numbers.
\end{problem}
%\begin{problem}
%Download the python code from \href{http://tlc.iith.ac.in/resources/tkcalc.py}{\url{http://tlc.iith.ac.in/resources/tkcalc.py}} and execute it.
%%\begin{lstlisting}
%%http://tlc.iith.ac.in/resources/tkcalc.py
%%\end{lstlisting}
%\end{problem}
\begin{problem}
Write all the required C routines for the calculator in Problem \ref{prob:calc} and generate the shared libraries. Test all the routines.
\end{problem}

\section{Shared libraries in Python}
%\lstinputlisting{./codes/commands}
\begin{problem}
Write a Python script to multiply two numbers using C function.
\end{problem}
\solution
\lstinputlisting{./codes/pythonprogs/mul.py}
\begin{problem}
Call the function written above in the Python GUI calculator to perform multipication.
\end{problem}

\solution
Download \textbf{calc\_mul\_root.py} file from the \cite{calc_mul_root} and save it in directory where \textbf{mul.c} is saved.  Execute  \textbf{calc\_mul\_root.py}.

\begin{problem}
Use C routines in \textbf{calc\_mul\_root.py} for all arithmetic operations in the calculator.
\end{problem}
\section{Integer Triangles}
\begin{problem}
Given the perimeter of a triangle (it should be an integer) write a C program to find all the possible triangles with integer sides.  
You just have to print the lengths of the sides of each such triangle.
\end{problem}
\begin{problem}
Create a GUI application in Python for the previous problem.
\end{problem}

\bibliography{IEEEabrv,hemanth_cpython.bib}
\end{document}

