\documentclass[journal,12pt,twocolumn]{IEEEtran}
%
\usepackage{setspace}
\usepackage{gensymb}
\usepackage{xcolor}
\usepackage{caption}
%\usepackage{subcaption}
%\doublespacing
\singlespacing

%\usepackage{graphicx}
%\usepackage{amssymb}
%\usepackage{relsize}
\usepackage[cmex10]{amsmath}
\usepackage{mathtools}
%\usepackage{amsthm}
%\interdisplaylinepenalty=2500
%\savesymbol{iint}
%\usepackage{txfonts}
%\restoresymbol{TXF}{iint}
%\usepackage{wasysym}
\usepackage{amsthm}
\usepackage{mathrsfs}
\usepackage{txfonts}
\usepackage{stfloats}
\usepackage{cite}
\usepackage{cases}
\usepackage{subfig}
%\usepackage{xtab}
\usepackage{longtable}
\usepackage{multirow}
%\usepackage{algorithm}
%\usepackage{algpseudocode}
\usepackage{enumitem}
\usepackage{mathtools}
\usepackage{hyperref}

%\usepackage[framemethod=tikz]{mdframed}
\usepackage{listings}
    \usepackage[latin1]{inputenc}                                 %%
    \usepackage{color}                                            %%
    \usepackage{array}                                            %%
    \usepackage{longtable}                                        %%
    \usepackage{calc}                                             %%
    \usepackage{multirow}                                         %%
    \usepackage{hhline}                                           %%
    \usepackage{ifthen}                                           %%
  %optionally (for landscape tables embedded in another document): %%
    \usepackage{lscape}     

\usepackage{iithtlc}
\usepackage{tikz}
\usepackage{circuitikz}
\usepackage{url}
\def\UrlBreaks{\do\/\do-}


%\usepackage{stmaryrd}


%\usepackage{wasysym}
%\newcounter{MYtempeqncnt}
\DeclareMathOperator*{\Res}{Res}
%\renewcommand{\baselinestretch}{2}
\renewcommand\thesection{\arabic{section}}
\renewcommand\thesubsection{\thesection.\arabic{subsection}}
\renewcommand\thesubsubsection{\thesubsection.\arabic{subsubsection}}

\renewcommand\thesectiondis{\arabic{section}}
\renewcommand\thesubsectiondis{\thesectiondis.\arabic{subsection}}
\renewcommand\thesubsubsectiondis{\thesubsectiondis.\arabic{subsubsection}}

% correct bad hyphenation here
\hyphenation{op-tical net-works semi-conduc-tor}

\lstset{
language=C,
frame=single, 
breaklines=true
}

%\lstset{
	%%basicstyle=\small\ttfamily\bfseries,
	%%numberstyle=\small\ttfamily,
	%language=Octave,
	%backgroundcolor=\color{white},
	%%frame=single,
	%%keywordstyle=\bfseries,
	%%breaklines=true,
	%%showstringspaces=false,
	%%xleftmargin=-10mm,
	%%aboveskip=-1mm,
	%%belowskip=0mm
%}

%\surroundwithmdframed[width=\columnwidth]{lstlisting}
\def\inputGnumericTable{}                                 %%
\lstset{
language=C,
frame=single, 
breaklines=true
}
 

\begin{document}
%

\theoremstyle{definition}
\newtheorem{theorem}{Theorem}[section]
\newtheorem{problem}{Problem}
\newtheorem{proposition}{Proposition}[section]
\newtheorem{lemma}{Lemma}[section]
\newtheorem{corollary}[theorem]{Corollary}
\newtheorem{example}{Example}[section]
\newtheorem{definition}{Definition}[section]
%\newtheorem{algorithm}{Algorithm}[section]
%\newtheorem{cor}{Corollary}
\newcommand{\BEQA}{\begin{eqnarray}}
\newcommand{\EEQA}{\end{eqnarray}}
\newcommand{\define}{\stackrel{\triangle}{=}}

\bibliographystyle{IEEEtran}
%\bibliographystyle{ieeetr}

\providecommand{\nCr}[2]{\,^{#1}C_{#2}} % nCr
\providecommand{\nPr}[2]{\,^{#1}P_{#2}} % nPr
\providecommand{\mbf}{\mathbf}
\providecommand{\pr}[1]{\ensuremath{\Pr\left(#1\right)}}
\providecommand{\qfunc}[1]{\ensuremath{Q\left(#1\right)}}
\providecommand{\sbrak}[1]{\ensuremath{{}\left[#1\right]}}
\providecommand{\lsbrak}[1]{\ensuremath{{}\left[#1\right.}}
\providecommand{\rsbrak}[1]{\ensuremath{{}\left.#1\right]}}
\providecommand{\brak}[1]{\ensuremath{\left(#1\right)}}
\providecommand{\lbrak}[1]{\ensuremath{\left(#1\right.}}
\providecommand{\rbrak}[1]{\ensuremath{\left.#1\right)}}
\providecommand{\cbrak}[1]{\ensuremath{\left\{#1\right\}}}
\providecommand{\lcbrak}[1]{\ensuremath{\left\{#1\right.}}
\providecommand{\rcbrak}[1]{\ensuremath{\left.#1\right\}}}
\theoremstyle{remark}
\newtheorem{rem}{Remark}
\newcommand{\sgn}{\mathop{\mathrm{sgn}}}
\providecommand{\abs}[1]{\left\vert#1\right\vert}
\providecommand{\res}[1]{\Res\displaylimits_{#1}} 
\providecommand{\norm}[1]{\lVert#1\rVert}
\providecommand{\mtx}[1]{\mathbf{#1}}
\providecommand{\mean}[1]{E\left[ #1 \right]}
\providecommand{\fourier}{\overset{\mathcal{F}}{ \rightleftharpoons}}
%\providecommand{\hilbert}{\overset{\mathcal{H}}{ \rightleftharpoons}}
\providecommand{\system}{\overset{\mathcal{H}}{ \longleftrightarrow}}
	%\newcommand{\solution}[2]{\textbf{Solution:}{#1}}
\newcommand{\solution}{\noindent \textbf{Solution: }}
\providecommand{\dec}[2]{\ensuremath{\overset{#1}{\underset{#2}{\gtrless}}}}
%\numberwithin{equation}{subsection}
\numberwithin{equation}{problem}
%\numberwithin{problem}{subsection}
%\numberwithin{definition}{subsection}
\makeatletter
\@addtoreset{figure}{problem}
\makeatother

\let\StandardTheFigure\thefigure
%\renewcommand{\thefigure}{\theproblem.\arabic{figure}}
\renewcommand{\thefigure}{\theproblem}


%\numberwithin{figure}{subsection}

%\numberwithin{equation}{subsection}
%\numberwithin{equation}{section}
%%\numberwithin{equation}{problem}
%%\numberwithin{problem}{subsection}
%\numberwithin{problem}{section}
%%\numberwithin{definition}{subsection}
%\makeatletter
%\@addtoreset{figure}{problem}
%\makeatother
%\makeatletter
%\@addtoreset{table}{problem}
%\makeatother

%\let\StandardTheFigure\thefigure
%\let\StandardTheTable\thetable
%%\renewcommand{\thefigure}{\theproblem.\arabic{figure}}
%\renewcommand{\thefigure}{\theproblem}
%\renewcommand{\thetable}{\theproblem}
%%\numberwithin{figure}{section}

%%\numberwithin{figure}{subsection}



\def\putbox#1#2#3{\makebox[0in][l]{\makebox[#1][l]{}\raisebox{\baselineskip}[0in][0in]{\raisebox{#2}[0in][0in]{#3}}}}
     \def\rightbox#1{\makebox[0in][r]{#1}}
     \def\centbox#1{\makebox[0in]{#1}}
     \def\topbox#1{\raisebox{-\baselineskip}[0in][0in]{#1}}
     \def\midbox#1{\raisebox{-0.5\baselineskip}[0in][0in]{#1}}

\vspace{3cm}

\title{ 
	\logo{
EE 1410: Data Structures
\\
Teaching Methodology
	}
}



% paper title
% can use linebreaks \\ within to get better formatting as desired
%\title{Matrix Analysis through Octave}
%
%
% author names and IEEE memberships
% note positions of commas and nonbreaking spaces ( ~ ) LaTeX will not break
% a structure at a ~ so this keeps an author's name from being broken across
% two lines.
% use \thanks{} to gain access to the first footnote area
% a separate \thanks must be used for each paragraph as LaTeX2e's \thanks
% was not built to handle multiple paragraphs
%

\author{G V V Sharma$^{*}$% <-this % stops a space
\thanks{*The author is with the Department
of Electrical Engineering, Indian Institute of Technology, Hyderabad
502285 India e-mail:  gadepall@iith.ac.in. 
%All content in this manual is released under GNU GPL.  Free and open source.
}% <-this % stops a space
%\thanks{J. Doe and J. Doe are with Anonymous University.}% <-this % stops a space
%\thanks{Manuscript received April 19, 2005; revised January 11, 2007.}}
}
% note the % following the last \IEEEmembership and also \thanks - 
% these prevent an unwanted space from occurring between the last author name
% and the end of the author line. i.e., if you had this:
% 
% \author{....lastname \thanks{...} \thanks{...} }
%                     ^------------^------------^----Do not want these spaces!
%
% a space would be appended to the last name and could cause every name on that
% line to be shifted left slightly. This is one of those "LaTeX things". For
% instance, "\textbf{A} \textbf{B}" will typeset as "A B" not "AB". To get
% "AB" then you have to do: "\textbf{A}\textbf{B}"
% \thanks is no different in this regard, so shield the last } of each \thanks
% that ends a line with a % and do not let a space in before the next \thanks.
% Spaces after \IEEEmembership other than the last one are OK (and needed) as
% you are supposed to have spaces between the names. For what it is worth,
% this is a minor point as most people would not even notice if the said evil
% space somehow managed to creep in.



% The paper headers
%\markboth{Journal of \LaTeX\ Class Files,~Vol.~6, No.~1, January~2007}%
%{Shell \MakeLowercase{\textit{et al.}}: Bare Demo of IEEEtran.cls for Journals}
% The only time the second header will appear is for the odd numbered pages
% after the title page when using the twoside option.
% 
% *** Note that you probably will NOT want to include the author's ***
% *** name in the headers of peer review papers.                   ***
% You can use \ifCLASSOPTIONpeerreview for conditional compilation here if
% you desire.




% If you want to put a publisher's ID mark on the page you can do it like
% this:
%\IEEEpubid{0000--0000/00\$00.00~\copyright~2007 IEEE}
% Remember, if you use this you must call \IEEEpubidadjcol in the second
% column for its text to clear the IEEEpubid mark.



% make the title area
\maketitle

\tableofcontents

\bigskip

\begin{abstract}
%\boldmath
This manual shows how  to introduce C and Python programming and data structures to beginners through a hands on approach. The philosophy of the
course is to teach students to use programming as a tool to solve problems.
\end{abstract}
\section{Requirements}
\begin{enumerate}
\item The course is of 2 credit and the hands on module should be completed in 30 hours.
\item Ideally, there should be no more than 40 students in the class.
\item Each student should have access to a  Linux computer and the TLC Arduino kit.
\item The hands on sessions should span over 10 days, with a 3 hour session every day.
\item At least 4 Teaching Assistants (TAs) are required.  These can be senior UG students or PG students.
\end{enumerate}
\section{Hands On Sessions}
Hands on sessions are conducted through the following manuals.
\begin{enumerate}
\item \cite{afe} introduces the concept of hardware and software.  This is done by using an arduino
to drive a seven segment display to obtain a decade counter.
\item \cite{gvv_cprog} introduces basic programming concepts like conditional statements, loops, functions and arrays.
This is done by using an arduino to drive a seven segment display to obtain a decade counter. The arduino IDE follows
the C language syntax, so the student learns how to program using C.
\item \cite{gvv_jee_2016_python} introduces Python by solving selected problems from the JEE mains mathematics paper
through Python programming.  For examples, problems related to coordinate geometry can be easily visualized through computations
using {\em numpy} and {\em scipy} and plotting through {\em matplotlib}. 
\item Through \cite{gvv_pythonc}, the student converts the computational part of the 
Python code in \cite{gvv_jee_2016_python} to C code and the output data is stored in files.  This data is then plotted
in Python.  This helps the student learn file handling in C.   
\item \cite{gvv_dstc} introduces the student to pointers, lists and trees through polynomial operations.
\end{enumerate}
\section{Evaluation}
\begin{enumerate}
\item \cite{gvv_jee_2017} is a collection of unsolved problems from the JEE 2017 mathematics mains paper.  The students
are expected to use the scripts in \cite{gvv_jee_2016_python} to generate their own python scripts for \cite{gvv_jee_2017}.
\item The next assignment for the students is to convert the python codes for \cite{gvv_jee_2017} to C.  
\item For data structures, each student is asked to write the code for polynomial addition and multiplication and another 
student is asked to document the syntax and logical errors and get the code running.  Through this approach, students
learn to read, write and debug code.
\item Multiple exams with equal weightage should be conducted on the above lines.

%progrbecome good programmers.
%explains Karnaugh maps and the finite state machine.  The Karnaugh map is explained through the truth tables in
%\cite{adld} and the decade counter in \cite{adld} is used for introducing the state machine.
%
%\item \cite{ee1110_gate} is a collection of problems in Boolean logic from past GATE papers which is shared with the students.  In the tutorial
%session, students are asked to solve the problems in \cite{ee1110_gate} in groups and the instructor and TAs help clear any doubts.
%The tutorial session is for 3 hours.
%\item In the exams, each student is given a different problem in \cite{ee1110_gate}.  The student has to solve the problem on paper and
%verify her result by programming the logic on the arduino.  
%\item At least 3 such exams are conducted to assess the student's learning. 
%Exams are conducted based on \cite{ee1110_gate}
%\href{http://tlc.iith.ac.in/img/gvv_a4s.pdf}{\url{http://tlc.iith.ac.in/img/gvv_a4s.pdf}}.  This manual 
\end{enumerate}
% IEEEtran.cls defaults to using nonbold math in the Abstract.
% This preserves the distinction between vectors and scalars. However,
% if the journal you are submitting to favors bold math in the abstract,
% then you can use LaTeX's standard command \boldmath at the very start
% of the abstract to achieve this. Many IEEE journals frown on math
% in the abstract anyway.

% Note that keywords are not normally used for peerreview papers.
%\begin{IEEEkeywords   }
%Cooperative diversity, decode and forward, piecewise linear
%\end{IEEEkeyword,m nbvxz 



% For peer review papers, you can put extra information on the cover
% page as needed:
% \ifCLASSOPTIONpeerreview
% \begin{center} \bfseries EDICS Category: 3-BBND \end{center}
% \fi
%
% For peerreview papers, this IEEEtran command inserts a page break and
% creates the second title. It will be ignored for other modes.
%\IEEEpeerreviewmaketitle


%\newpage
%\section{Component Pin Diagrams}
%%
%\renewcommand{\theequation}{\theenumi}
\begin{enumerate}[label=\thesection.\arabic*.,ref=\thesection.\theenumi]
\numberwithin{equation}{enumi}
\item Let the medians $BE$ and $CF$ in Fig. \ref{fig:3.12.3_ch1_two_median} intersect at $O$, such that
\begin{equation}
\begin{split}
\frac{OB}{OE} &= k_1
\\
\frac{OC}{OF} &= k_2
\end{split}
\end{equation}
%Then  $k_1 = k_2 = 2$.
%
\begin{figure}[!h]
\centering
\resizebox {\columnwidth} {!} {
\begin{tikzpicture}
  [
    scale=2,
    >=stealth,
    point/.style = {draw, circle,  fill = black, inner sep = 0.5pt},
    dot/.style   = {draw, circle,  fill = black, inner sep = .2pt},
  ]
  \coordinate [point, label={below left:$B$ $(0, 0)$}] (B) at (0, 0);
    \node (A) at +(60:{2*sqrt(3)}) [point, label = above:$A$ ${(a,b)}$  ] {};
  \coordinate [point, label={below left:$(c,0)$ $C$ }] (C) at ($ (3,0) + sqrt(3)*(1,0) $);

  \draw  (A) -- (C) -- (B) -- (A);
  \node (E) at ($(A)!0.5!(C)$) [point, label = {right:$E$}]{};
  \node (F) at ($(A)!0.5!(B)$) [point, label = {left:$F$}]{};
  \path
     (B)    edge  node[sloped, anchor=center, below, text width=2.0cm] { $k_1:1$}     (E)  
	 (C)    edge  node[sloped, anchor=east, below, text width=2.0cm] { $1:k_2$}     (F);
  \node (O) at ($(B)!0.67!(E)$) [point, label = {below:$O$}]{};  
\end{tikzpicture}


}
\caption{Medians $BE$ and $CF$}
\label{fig:3.12.3_ch1_two_median}
\end{figure}
%Let the coordinates of $A$, $B$ and $C$ be $\brak{a,b}$, $\brak{0,0}$ and $\brak{c,0}$ respectively. 
Using \eqref{eq:line_section_form},
%
\begin{align}
E &= \frac{\vec{A}+\vec{B}}{2} 
\\
F &= \frac{\vec{A}+\vec{C}}{2} 
\label{eq:3.12.3_ch1_ratio_ef}
\end{align}
%
Similarly, since $O$ divides $BE$ in the ratio $k_1:1$ and $CF$ in the ratio $k_2:1$.
 %
\begin{align}
O = \frac{k_1\vec{E}+\vec{B}}{k_1+1} &=  \frac{k_2\vec{F}+\vec{C}}{k_2+1} 
\\
\implies \frac{k_1\brak{\frac{\vec{A}+\vec{B}}{2}} +B}{k_1+1} &=  \frac{k_2\brak{\frac{\vec{A}+\vec{C}}{2} }+C}{k_2+1} 
\label{eq:3.12.3_ch1_ratio_2}
\end{align}
upon substituting from \eqref{eq:3.12.3_ch1_ratio_ef}.
Simplifying \eqref{eq:3.12.3_ch1_ratio_2},
\begin{align}
\frac{k_1\brak{\vec{A}+\vec{C}} +2\vec{B}}{k_1+1} =  \frac{k_2\brak{\vec{A}+\vec{B}}+2\vec{C}}{k_2+1} 
\end{align}
which can be expressed as
\begin{multline}
\implies \sbrak{k_1\brak{k_2+1}-k_2\brak{k_1+1}}\vec{A}
\\
 +\sbrak{2\brak{k_2+1}-k_2\brak{k_1+1}}\vec{B}
\\ +  \sbrak{k_1\brak{k_2+1} -2\brak{k_1+1}}\vec{C} = 0
\end{multline}
resulting in 
\begin{align}
\vec{B} = \frac{\brak{k_1-k_2}\vec{A}+\brak{k_1k_2 -k_1 -2}}{k_1k_2 -k_2 -2}
\end{align}
%
If the above equation has a solution, then $\vec{A}, \vec{B}$ and $\vec{C}$ lie on a straight line.  Since that is not the case, the only possibility is 
\begin{align}
k_1-k_2 &= 0
\\
k_1k_2 -k_1 -2 &= 0
\\
k_1k_2 -k_2 -2 &= 0
\\
\implies k_1=k_2&=2
\end{align}
\item In Fig. \ref{fig:3.12.3_ch1_two_median},
\begin{align}
\vec{E} &=  \frac{\vec{A}+\vec{C}}{2} \quad \text{and}
\\
\vec{O}&= \frac{\vec{B}+2\vec{E}}{3}
\\
&= \frac{\vec{A}+\vec{B}+\vec{C}}{3}
\end{align}
\end{enumerate}
	

%

%\newpage
%\section{Display Control through Hardware }
%\renewcommand{\theequation}{\theenumi}
\begin{enumerate}[label=\thesection.\arabic*.,ref=\thesection.\theenumi]
\numberwithin{equation}{enumi}
\item Let the medians $BE$ and $CF$ in Fig. \ref{fig:3.12.3_ch1_two_median} intersect at $O$, such that
\begin{equation}
\begin{split}
\frac{OB}{OE} &= k_1
\\
\frac{OC}{OF} &= k_2
\end{split}
\end{equation}
%Then  $k_1 = k_2 = 2$.
%
\begin{figure}[!h]
\centering
\resizebox {\columnwidth} {!} {
\begin{tikzpicture}
  [
    scale=2,
    >=stealth,
    point/.style = {draw, circle,  fill = black, inner sep = 0.5pt},
    dot/.style   = {draw, circle,  fill = black, inner sep = .2pt},
  ]
  \coordinate [point, label={below left:$B$ $(0, 0)$}] (B) at (0, 0);
    \node (A) at +(60:{2*sqrt(3)}) [point, label = above:$A$ ${(a,b)}$  ] {};
  \coordinate [point, label={below left:$(c,0)$ $C$ }] (C) at ($ (3,0) + sqrt(3)*(1,0) $);

  \draw  (A) -- (C) -- (B) -- (A);
  \node (E) at ($(A)!0.5!(C)$) [point, label = {right:$E$}]{};
  \node (F) at ($(A)!0.5!(B)$) [point, label = {left:$F$}]{};
  \path
     (B)    edge  node[sloped, anchor=center, below, text width=2.0cm] { $k_1:1$}     (E)  
	 (C)    edge  node[sloped, anchor=east, below, text width=2.0cm] { $1:k_2$}     (F);
  \node (O) at ($(B)!0.67!(E)$) [point, label = {below:$O$}]{};  
\end{tikzpicture}


}
\caption{Medians $BE$ and $CF$}
\label{fig:3.12.3_ch1_two_median}
\end{figure}
%Let the coordinates of $A$, $B$ and $C$ be $\brak{a,b}$, $\brak{0,0}$ and $\brak{c,0}$ respectively. 
Using \eqref{eq:line_section_form},
%
\begin{align}
E &= \frac{\vec{A}+\vec{B}}{2} 
\\
F &= \frac{\vec{A}+\vec{C}}{2} 
\label{eq:3.12.3_ch1_ratio_ef}
\end{align}
%
Similarly, since $O$ divides $BE$ in the ratio $k_1:1$ and $CF$ in the ratio $k_2:1$.
 %
\begin{align}
O = \frac{k_1\vec{E}+\vec{B}}{k_1+1} &=  \frac{k_2\vec{F}+\vec{C}}{k_2+1} 
\\
\implies \frac{k_1\brak{\frac{\vec{A}+\vec{B}}{2}} +B}{k_1+1} &=  \frac{k_2\brak{\frac{\vec{A}+\vec{C}}{2} }+C}{k_2+1} 
\label{eq:3.12.3_ch1_ratio_2}
\end{align}
upon substituting from \eqref{eq:3.12.3_ch1_ratio_ef}.
Simplifying \eqref{eq:3.12.3_ch1_ratio_2},
\begin{align}
\frac{k_1\brak{\vec{A}+\vec{C}} +2\vec{B}}{k_1+1} =  \frac{k_2\brak{\vec{A}+\vec{B}}+2\vec{C}}{k_2+1} 
\end{align}
which can be expressed as
\begin{multline}
\implies \sbrak{k_1\brak{k_2+1}-k_2\brak{k_1+1}}\vec{A}
\\
 +\sbrak{2\brak{k_2+1}-k_2\brak{k_1+1}}\vec{B}
\\ +  \sbrak{k_1\brak{k_2+1} -2\brak{k_1+1}}\vec{C} = 0
\end{multline}
resulting in 
\begin{align}
\vec{B} = \frac{\brak{k_1-k_2}\vec{A}+\brak{k_1k_2 -k_1 -2}}{k_1k_2 -k_2 -2}
\end{align}
%
If the above equation has a solution, then $\vec{A}, \vec{B}$ and $\vec{C}$ lie on a straight line.  Since that is not the case, the only possibility is 
\begin{align}
k_1-k_2 &= 0
\\
k_1k_2 -k_1 -2 &= 0
\\
k_1k_2 -k_2 -2 &= 0
\\
\implies k_1=k_2&=2
\end{align}
\item In Fig. \ref{fig:3.12.3_ch1_two_median},
\begin{align}
\vec{E} &=  \frac{\vec{A}+\vec{C}}{2} \quad \text{and}
\\
\vec{O}&= \frac{\vec{B}+2\vec{E}}{3}
\\
&= \frac{\vec{A}+\vec{B}+\vec{C}}{3}
\end{align}
\end{enumerate}
	

%%
%\section{Display Control through Arduino Software}
%\input{./chapters/chapter2}
%%%
%\section{Decade Counter through Arduino}
%\input{./chapters/chapter3}
%%
%\section{Karnaugh Maps}
%\input{./chapters/chapter4}
%%
%\section{Sequential Logic}
%\input{./chapters/chapter5}
%
%\section{C Programming}
%\input{./chapters/chapter6}

%%%%%%%%%%%%%%%%%%%%%%%%%%%%%%%%%%%%%%%%%%%%%%%%%%%%%%%%%%%%%%%%%%%%%%%
%%                                                                  %%
%%  This is the header of a LaTeX2e file exported from Gnumeric.    %%
%%                                                                  %%
%%  This file can be compiled as it stands or included in another   %%
%%  LaTeX document. The table is based on the longtable package so  %%
%%  the longtable options (headers, footers...) can be set in the   %%
%%  preamble section below (see PRAMBLE).                           %%
%%                                                                  %%
%%  To include the file in another, the following two lines must be %%
%%  in the including file:                                          %%
%%        \def\inputGnumericTable{}                                 %%
%%  at the beginning of the file and:                               %%
%%        \input{name-of-this-file.tex}                             %%
%%  where the table is to be placed. Note also that the including   %%
%%  file must use the following packages for the table to be        %%
%%  rendered correctly:                                             %%
%%    \usepackage[latin1]{inputenc}                                 %%
%%    \usepackage{color}                                            %%
%%    \usepackage{array}                                            %%
%%    \usepackage{longtable}                                        %%
%%    \usepackage{calc}                                             %%
%%    \usepackage{multirow}                                         %%
%%    \usepackage{hhline}                                           %%
%%    \usepackage{ifthen}                                           %%
%%  optionally (for landscape tables embedded in another document): %%
%%    \usepackage{lscape}                                           %%
%%                                                                  %%
%%%%%%%%%%%%%%%%%%%%%%%%%%%%%%%%%%%%%%%%%%%%%%%%%%%%%%%%%%%%%%%%%%%%%%



%%  This section checks if we are begin input into another file or  %%
%%  the file will be compiled alone. First use a macro taken from   %%
%%  the TeXbook ex 7.7 (suggestion of Han-Wen Nienhuys).            %%
\def\ifundefined#1{\expandafter\ifx\csname#1\endcsname\relax}


%%  Check for the \def token for inputed files. If it is not        %%
%%  defined, the file will be processed as a standalone and the     %%
%%  preamble will be used.                                          %%
\ifundefined{inputGnumericTable}

%%  We must be able to close or not the document at the end.        %%
	\def\gnumericTableEnd{\end{document}}


%%%%%%%%%%%%%%%%%%%%%%%%%%%%%%%%%%%%%%%%%%%%%%%%%%%%%%%%%%%%%%%%%%%%%%
%%                                                                  %%
%%  This is the PREAMBLE. Change these values to get the right      %%
%%  paper size and other niceties.                                  %%
%%                                                                  %%
%%%%%%%%%%%%%%%%%%%%%%%%%%%%%%%%%%%%%%%%%%%%%%%%%%%%%%%%%%%%%%%%%%%%%%

	\documentclass[12pt%
			  %,landscape%
                    ]{report}
       \usepackage[latin1]{inputenc}
       \usepackage{fullpage}
       \usepackage{color}
       \usepackage{array}
       \usepackage{longtable}
       \usepackage{calc}
       \usepackage{multirow}
       \usepackage{hhline}
       \usepackage{ifthen}

	\begin{document}


%%  End of the preamble for the standalone. The next section is for %%
%%  documents which are included into other LaTeX2e files.          %%
\else

%%  We are not a stand alone document. For a regular table, we will %%
%%  have no preamble and only define the closing to mean nothing.   %%
    \def\gnumericTableEnd{}

%%  If we want landscape mode in an embedded document, comment out  %%
%%  the line above and uncomment the two below. The table will      %%
%%  begin on a new page and run in landscape mode.                  %%
%       \def\gnumericTableEnd{\end{landscape}}
%       \begin{landscape}


%%  End of the else clause for this file being \input.              %%
\fi

%%%%%%%%%%%%%%%%%%%%%%%%%%%%%%%%%%%%%%%%%%%%%%%%%%%%%%%%%%%%%%%%%%%%%%
%%                                                                  %%
%%  The rest is the gnumeric table, except for the closing          %%
%%  statement. Changes below will alter the table's appearance.     %%
%%                                                                  %%
%%%%%%%%%%%%%%%%%%%%%%%%%%%%%%%%%%%%%%%%%%%%%%%%%%%%%%%%%%%%%%%%%%%%%%

\providecommand{\gnumericmathit}[1]{#1} 
%%  Uncomment the next line if you would like your numbers to be in %%
%%  italics if they are italizised in the gnumeric table.           %%
%\renewcommand{\gnumericmathit}[1]{\mathit{#1}}
\providecommand{\gnumericPB}[1]%
{\let\gnumericTemp=\\#1\let\\=\gnumericTemp\hspace{0pt}}
 \ifundefined{gnumericTableWidthDefined}
        \newlength{\gnumericTableWidth}
        \newlength{\gnumericTableWidthComplete}
        \newlength{\gnumericMultiRowLength}
        \global\def\gnumericTableWidthDefined{}
 \fi
%% The following setting protects this code from babel shorthands.  %%
 \ifthenelse{\isundefined{\languageshorthands}}{}{\languageshorthands{english}}
%%  The default table format retains the relative column widths of  %%
%%  gnumeric. They can easily be changed to c, r or l. In that case %%
%%  you may want to comment out the next line and uncomment the one %%
%%  thereafter                                                      %%
\providecommand\gnumbox{\makebox[0pt]}
%%\providecommand\gnumbox[1][]{\makebox}

%% to adjust positions in multirow situations                       %%
\setlength{\bigstrutjot}{\jot}
\setlength{\extrarowheight}{\doublerulesep}

%%  The \setlongtables command keeps column widths the same across  %%
%%  pages. Simply comment out next line for varying column widths.  %%
\setlongtables

\setlength\gnumericTableWidth{%
	50pt+%
	71pt+%
	50pt+%
	50pt+%
	50pt+%
	50pt+%
	50pt+%
	50pt+%
0pt}
\def\gumericNumCols{8}
\setlength\gnumericTableWidthComplete{\gnumericTableWidth+%
         \tabcolsep*\gumericNumCols*2+\arrayrulewidth*\gumericNumCols}
\ifthenelse{\lengthtest{\gnumericTableWidthComplete > \linewidth}}%
         {\def\gnumericScale{\ratio{\linewidth-%
                        \tabcolsep*\gumericNumCols*2-%
                        \arrayrulewidth*\gumericNumCols}%
{\gnumericTableWidth}}}%
{\def\gnumericScale{1}}

%%%%%%%%%%%%%%%%%%%%%%%%%%%%%%%%%%%%%%%%%%%%%%%%%%%%%%%%%%%%%%%%%%%%%%
%%                                                                  %%
%% The following are the widths of the various columns. We are      %%
%% defining them here because then they are easier to change.       %%
%% Depending on the cell formats we may use them more than once.    %%
%%                                                                  %%
%%%%%%%%%%%%%%%%%%%%%%%%%%%%%%%%%%%%%%%%%%%%%%%%%%%%%%%%%%%%%%%%%%%%%%

\ifthenelse{\isundefined{\gnumericColA}}{\newlength{\gnumericColA}}{}\settowidth{\gnumericColA}{\begin{tabular}{@{}p{45pt*\gnumericScale}@{}}x\end{tabular}}
\ifthenelse{\isundefined{\gnumericColB}}{\newlength{\gnumericColB}}{}\settowidth{\gnumericColB}{\begin{tabular}{@{}p{45pt*\gnumericScale}@{}}x\end{tabular}}
\ifthenelse{\isundefined{\gnumericColC}}{\newlength{\gnumericColC}}{}\settowidth{\gnumericColC}{\begin{tabular}{@{}p{45pt*\gnumericScale}@{}}x\end{tabular}}
\ifthenelse{\isundefined{\gnumericColD}}{\newlength{\gnumericColD}}{}\settowidth{\gnumericColD}{\begin{tabular}{@{}p{45pt*\gnumericScale}@{}}x\end{tabular}}
\ifthenelse{\isundefined{\gnumericColE}}{\newlength{\gnumericColE}}{}\settowidth{\gnumericColE}{\begin{tabular}{@{}p{45pt*\gnumericScale}@{}}x\end{tabular}}
\ifthenelse{\isundefined{\gnumericColF}}{\newlength{\gnumericColF}}{}\settowidth{\gnumericColF}{\begin{tabular}{@{}p{45pt*\gnumericScale}@{}}x\end{tabular}}
\ifthenelse{\isundefined{\gnumericColG}}{\newlength{\gnumericColG}}{}\settowidth{\gnumericColG}{\begin{tabular}{@{}p{45pt*\gnumericScale}@{}}x\end{tabular}}
\ifthenelse{\isundefined{\gnumericColH}}{\newlength{\gnumericColH}}{}\settowidth{\gnumericColH}{\begin{tabular}{@{}p{80pt*\gnumericScale}@{}}x\end{tabular}}

\begin{table}[!h]
\begin{tabular}{
%\begin{longtable}[c]{%
	b{\gnumericColA}%
	b{\gnumericColB}%
	b{\gnumericColC}%
	b{\gnumericColD}%
	b{\gnumericColE}%
	b{\gnumericColF}%
	b{\gnumericColG}%
	b{\gnumericColH}%
	}

%%%%%%%%%%%%%%%%%%%%%%%%%%%%%%%%%%%%%%%%%%%%%%%%%%%%%%%%%%%%%%%%%%%%%%
%%  The longtable options. (Caption, headers... see Goosens, p.124) %%
%	\caption{The Table Caption.}             \\	%
% \hline	% Across the top of the table.
%%  The rest of these options are table rows which are placed on    %%
%%  the first, last or every page. Use \multicolumn if you want.    %%

%%  Header for the first page.                                      %%
%	\multicolumn{8}{c}{The First Header} \\ \hline 
%	\multicolumn{1}{c}{colTag}	%Column 1
%	&\multicolumn{1}{c}{colTag}	%Column 2
%	&\multicolumn{1}{c}{colTag}	%Column 3
%	&\multicolumn{1}{c}{colTag}	%Column 4
%	&\multicolumn{1}{c}{colTag}	%Column 5
%	&\multicolumn{1}{c}{colTag}	%Column 6
%	&\multicolumn{1}{c}{colTag}	%Column 7
%	&\multicolumn{1}{c}{colTag}	\\ \hline %Last column
%	\endfirsthead

%%  The running header definition.                                  %%
%	\hline
%	\multicolumn{8}{l}{\ldots\small\slshape continued} \\ \hline
%	\multicolumn{1}{c}{colTag}	%Column 1
%	&\multicolumn{1}{c}{colTag}	%Column 2
%	&\multicolumn{1}{c}{colTag}	%Column 3
%	&\multicolumn{1}{c}{colTag}	%Column 4
%	&\multicolumn{1}{c}{colTag}	%Column 5
%	&\multicolumn{1}{c}{colTag}	%Column 6
%	&\multicolumn{1}{c}{colTag}	%Column 7
%	&\multicolumn{1}{c}{colTag}	\\ \hline %Last column
%	\endhead

%%  The running footer definition.                                  %%
%	\hline
%	\multicolumn{8}{r}{\small\slshape continued\ldots} \\
%	\endfoot

%%  The ending footer definition.                                   %%
%	\multicolumn{8}{c}{That's all folks} \\ \hline 
%	\endlastfoot
%%%%%%%%%%%%%%%%%%%%%%%%%%%%%%%%%%%%%%%%%%%%%%%%%%%%%%%%%%%%%%%%%%%%%%

\hhline{|-|-|-|-|-|-|-|-}
	 \multicolumn{1}{|p{\gnumericColA}|}%
	{\gnumericPB{\centering}\gnumbox{a}}
	&\multicolumn{1}{p{\gnumericColB}|}%
	{\gnumericPB{\centering}\gnumbox{b}}
	&\multicolumn{1}{p{\gnumericColC}|}%
	{\gnumericPB{\centering}\gnumbox{c}}
	&\multicolumn{1}{p{\gnumericColD}|}%
	{\gnumericPB{\centering}\gnumbox{d}}
	&\multicolumn{1}{p{\gnumericColE}|}%
	{\gnumericPB{\centering}\gnumbox{e}}
	&\multicolumn{1}{p{\gnumericColF}|}%
	{\gnumericPB{\centering}\gnumbox{f}}
	&\multicolumn{1}{p{\gnumericColG}|}%
	{\gnumericPB{\centering}\gnumbox{g}}
	&\multicolumn{1}{p{\gnumericColH}|}%
	{\gnumericPB{\centering}\gnumbox{decimal}}
\\
\hhline{|--------|}
	 \multicolumn{1}{|p{\gnumericColA}|}%
	{\gnumericPB{\centering}\gnumbox{1}}
	&\multicolumn{1}{p{\gnumericColB}|}%
	{\gnumericPB{\centering}\gnumbox{0}}
	&\multicolumn{1}{p{\gnumericColC}|}%
	{\gnumericPB{\centering}\gnumbox{0}}
	&\multicolumn{1}{p{\gnumericColD}|}%
	{\gnumericPB{\centering}\gnumbox{1}}
	&\multicolumn{1}{p{\gnumericColE}|}%
	{\gnumericPB{\centering}\gnumbox{1}}
	&\multicolumn{1}{p{\gnumericColF}|}%
	{\gnumericPB{\centering}\gnumbox{1}}
	&\multicolumn{1}{p{\gnumericColG}|}%
	{\gnumericPB{\centering}\gnumbox{1}}
	&\multicolumn{1}{p{\gnumericColH}|}%
	{\gnumericPB{\centering}\gnumbox{1}}
\\
\hhline{|-|-|-|-|-|-|-|-|}
%\end{longtable}
\end{tabular}
\caption{}
\label{table:arduioport}
\end{table}

\ifthenelse{\isundefined{\languageshorthands}}{}{\languageshorthands{\languagename}}
\gnumericTableEnd


\bibliography{IEEEabrv,gvv_ee1410_method}

%\input{chapter2} 
%%
%\newpage
%\section{$M$-ary Modulation}
%\input{chapter3} 
%
%\newpage
%\section{BER in Rayleigh Flat Slowly Fading Channels}
%\input{chapter4} 

\end{document}


